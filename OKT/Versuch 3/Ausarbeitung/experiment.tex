\chapter{Experiment}

In the Experiment an austrian one cent coin was examined. The goal of the experiment is to understand the principle of the OCT and get to know the usage of the system.

\section{Start-up of the OCT system}

To start the OCT-system first the measurement computer, the scanning mirror drivers, the detector unit and the Laser has to be switched on. The laser is operational if the LED labled 'ready' is on. At the measurement computer the LabView program 'OCT\_student.vi' is started.

To check the functionality of the system a 'continous scan' is performed. The current plots of the A-scan in k-space and z-space are shown in the LabView window. In the A-scan in z-space a peak is visible. This represents the position of the surface of the sample. The position of the sample can be moved by a positioning screw. If the position of the peak is moved below zero, another peak becomes visible moving in positive direction. Thus the modulation of the A-scan in k-space consists of two symetrical peaks in z-space. Since k-space and z-space are related by a fourier transform (cf. \ref{sec:fourier}), the modulation carrying the information must be sinusodial. \comseb{versteht man das?}


\section{Performing a 3D-scan of the sample}