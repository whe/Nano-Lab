\chapter{Preparation}
\section{Loss Mechanisms in Optical Fibers}

The guiding mechanism of a mode in an optical fiber can be described by an electromagnetic wave model. A wave hitting a dielectic boundary at a certain angle is reflected completely. For a waveguide structure like shown in figure \ref{fig:slab1} the wave is confined to the core, when the incident angle $\theta$ is smaller than the angle of total reflection. But since the Fields in the waveguide propagate as waves an additional condition has to be fulfilled. The x component of the Fields has to be a standing wave, because else it would deplete itself. Thus the wavenumber in the core in x direction times the height of the core $k\i{1x}h$ plus the phase jump caused by the reflections on the boundarys has to be an integer multiple of $\pi$:
\begin{equation}
 -2k\i{1x}h+2\varphi = -m\cdot2\pi
\end{equation}
This leads to several guided modes in a waveguide structure, dependent on the height of the core and the used materials. The described principle can be easyly transferred to cylindrically symetric structures like fibers or even more complex geometries.

The losses in optical fibers can be explained by different mechamisms. First there is the intrinsic material absorbtion. From the Kramers-Kronig relation of the real and imaginary part of the dielectric susceptibility it can be derived, that, if the attenuation of a medium is zero ($\chi\i{i}=0$ the medium is also dispersionless ($\chi=0$). Thus every dispersive medium has absorption.

Second there are the extrinsic losses in the fiber. These are caused by material impurities in the material like metal or OH$^-$ ions.
\todo{reighley scattering}
\todo{FIBER LOSSES - OWF Slides 184 ff.}



\section{Optical Time Domain Reflectometry}

\section{Characterisation of Passive Optical Networks}