\chapter{Preparation}
\section{Loss Mechanisms in Optical Fibers}

\subsection{Principles of Wave Guiding in Optical FIbers}
The guiding mechanism of a mode in an optical fiber can be described by an electromagnetic wave model. A wave hitting a dielectic boundary at a certain angle is reflected completely. For a waveguide structure like shown in figure \ref{fig:slab1} the wave is confined to the core, when the incident angle $\theta$ is smaller than the angle of total reflection. But since the Fields in the waveguide propagate as waves an additional condition has to be fulfilled. The x component of the Fields has to be a standing wave, because else it would deplete itself. Thus the wavenumber in the core in x direction times the height of the core $k\i{1x}h$ plus the phase jump caused by the reflections on the boundarys has to be an integer multiple of $\pi$:
\begin{equation}
 -2k\i{1x}h+2\varphi = -m\cdot2\pi
\end{equation}
This leads to several guided modes in a waveguide structure, dependent on the height of the core and the used materials. The described principle can be easyly transferred to cylindrically symetric structures like fibers or even more complex geometries.

\subsection{Loss Mechanisms in Optical Fibers}
\label{loss}
The losses in optical fibers can be explained by different mechanisms. First there is the intrinsic material absorption. From the Kramers-Kronig relation of the real and imaginary part of the dielectric susceptibility it can be derived, that, if the attenuation of a medium is zero ($\chi\i{i}=0$ the medium is also dispersionless ($\chi=0$). Thus every dispersive medium has absorption. Furthermore there are the extrinsic losses in the fiber. These are caused by material impurities in the material like metal or OH$^-$ ions. Another source of losses is the Rayleigh scattering. This scattering is caused by random index fluctuations smaller than a wavelength. These index fluctuations arise during the freezing of the silica molecules during the fiber fabrication. 

\subsection{Bends in Optical Fibers}

When a waveguide is bent it shows significant loss due to radiation. A wavefront propagates with its group velocity. If the waveguide is bent the wavefront at the inner boundary travels further in the same time than the wavefront at the outer boundary. This leads to a bigger part of the field that is not confined to the waveguide core. If the waveguide is bent further the fiber can break.


\section{Optical Time Domain Reflectometry}
Optical time domain reflectometry (OTDR) is a methodology to measure and characterize optical networks. It is possible to estimate losses in the fiber caused by connections or damages of the fiber. The attenuation of the fiber can be estimated just as the fiber length.

\subsection{Measurement principles of OTDR}
To analyze an optical network wit OTDR an optical time domain reflectometer sends in a short laser pulse with a certain amount of energy. The pulse travels trough the fiber and gets reflected and scattered at fiber connections, switches, damages and impurities in the fiber. 

After sending the pulse into the fiber the reflectometer starts listening on the port. An directional coupler leads the incoming light to a photodetector. Using the photodetector backscattered light can be detected. The time when the backscattered signal comes in is used to calculated the distance of the location the light got scattered.
There are two kinds of reflections that are important for the OTDR.
The Fresnel reflection is caused by an index contrast. This is the case at a splice, an open connection or a damaged fiber. Figure \ref{fig:fresnel}\footnotemark[1] shows different Fresnel reflections that can be measured. Empirical knowledge of the reflections of different connection and events gives information about the detected reflection.
 

\begin{figure}%
\centering
\includegraphics[width=.6\columnwidth]{grafiken/fresnel.png}%
\caption{Different Fresnel reflection caused by by (1) mechanical splices, (2) bulkhead and (3) open connection}%
\label{fig:fresnel}%
\end{figure}

The other scattering mechanism is the Rayleigh scattering (cf. \ref{loss}). The result of Rayleigh scattering is a straight slope in the OTDR trace. With that information the attenuation of the fiber can be calculated.\footnote[1]{EXFO - Application Note 194}





\


\section{Characterisation of Passive Optical Networks}

\begin{figure}%
\includegraphics[width=\columnwidth]{grafiken/gpon.jpg}%
\caption{GPON Network}%
\label{}%
\end{figure}

A standard PON Network has three main components - OLT, Splitter and ONP.

It uses Wavelength Division Multiplexing (WDM) to differentiate the upstream ($\lambda$~=~1310~nm) from downstream ($\lambda~=~$1490~nm).

The OLT (Optical Line Terminal) provides the interface between the PON and the service provider's network services. These typically include:

\begin{itemize}
	\item Internet Protocol (IP) traffic over gigabit/s, 10 Gbit/s, or 100 Mbit/s Ethernet
\item standard time division multiplexed (TDM) interfaces such as SONET or SDH
\item ATM UNI at 155-622 Mbit/s
\end{itemize}

Splitters

Although both the GPON and EPON protocols permit large split ratios (up to 128 subscribers for GPON, up to 32,768 for EPON), in practice most PONs are deployed with a split ratio of 1x32 or smaller


ONTs

Optical Network Transmitters (ONTs) a.k.a. Optical Network Units (ONUs) are devices that transforms incoming optical signals into electronics at a customer's premises in order to provide telecommunications services over the optical network.
\begin{figure}%
\centering
\includegraphics[width=.4\columnwidth]{grafiken/onu.jpg}%
\caption{Optical Network Transmitter}%
\label{}%
\end{figure}

There are different methods for characterizing the optical networks. Examples include testing all points from the central office (CO) to the optical network terminal (ONT), testing only some parts of the network. Still the OTDR method has proved itself to be most applicable as it is a single-ended method which to lead to great advantages of the network providers (reduced staff time, lower costs).

Two main different kinds of OTDR are most commonly used - standard OTDR and PON-Optimized OTDR.

By OTDR the link is measured from the ONT to the OLT and distortion in the link is observed. A small pulse with $\lambda$ = 1625 nm is sent through the fiber. Losses in the backscattered signal are proof for bad connection (a bend in the fiber for example).

As the standard OTDR results can be significantly distorted by many factors which is why a PON Optimized OTDR is used. With the PON-optimized OTDR, distortion after the splitter drop is greatly reduced, and the result is highly repeatable and reliable. The user can measure the loss of the splitter and the cumulative link loss, as well as identify whether any unexpected physical event occurred before or after the splitter.








