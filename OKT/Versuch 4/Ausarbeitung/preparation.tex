\chapter{Preparation}
\section{Loss Mechanisms in Optical Fibers}

\subsection{Principles of Wave Guiding in Optical FIbers}
The guiding mechanism of a mode in an optical fiber can be described by an electromagnetic wave model. A wave hitting a dielectic boundary at a certain angle is reflected completely. For a waveguide structure like shown in figure \ref{fig:slab1} the wave is confined to the core, when the incident angle $\theta$ is smaller than the angle of total reflection. But since the Fields in the waveguide propagate as waves an additional condition has to be fulfilled. The x component of the Fields has to be a standing wave, because else it would deplete itself. Thus the wavenumber in the core in x direction times the height of the core $k\i{1x}h$ plus the phase jump caused by the reflections on the boundarys has to be an integer multiple of $\pi$:
\begin{equation}
 -2k\i{1x}h+2\varphi = -m\cdot2\pi
\end{equation}
This leads to several guided modes in a waveguide structure, dependent on the height of the core and the used materials. The described principle can be easyly transferred to cylindrically symetric structures like fibers or even more complex geometries.

\subsection{Loss Mechanisms in Optical Fibers}
The losses in optical fibers can be explained by different mechamisms. First there is the intrinsic material absorbtion. From the Kramers-Kronig relation of the real and imaginary part of the dielectric susceptibility it can be derived, that, if the attenuation of a medium is zero ($\chi\i{i}=0$ the medium is also dispersionless ($\chi=0$). Thus every dispersive medium has absorption. Furthermore there are the extrinsic losses in the fiber. These are caused by material impurities in the material like metal or OH$^-$ ions. Another source of losses is the rayleigh scattering. This scattering is caused by random index fluctuations smaller than a wavelength. These index fluctuations arise during the freezing of the silica molecules during the fiber fabrication. 

\subsection{Bends in Optical Fibers}

When a waveguide is bent it shows significant loss due to radiation. A wavefront propagates with its group velocity. If the waveguide is bent the wavefront at the inner boundary travels further in the same time than the wavefront at the outer boundary. This leads to a bigger part of the field that is not confined to the waveguide core and therefore suffers more attenuation. If the waveguide is bent further the fiber can crack.


\section{Optical Time Domain Reflectometry}



\subsection{Measurement Range of an OTDR}
The dynamic range is the main limiting factor for the reach of an OTDR. To estimate the measurement range of an OTDR the noise level and attenuation like caused by fiber connectors, bends and splices have to be taken into account. As a rule of thumb the dynamic range of the OTDR should be 5 to 8 dB higher than the loss in the fiber.
Another limiting factor for the range of the OTDR is the pulse width. The shorter a pulse is, the less energy it carries. Through the attenuation of the fiber the pulse Amplitude of the pulse decreases and thus drops faster beneath the noise level if the pulse is short.



  

\section{Characterisation of Passive Optical Networks}