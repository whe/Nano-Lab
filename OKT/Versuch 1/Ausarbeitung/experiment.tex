 \chapter{Experimental Tasks}

\begin{figure}[ht]
\centering
\includegraphics[width=.6\columnwidth]{Grafiken/measurement1.pdf}%
\caption{}%
\label{fig:measurement1}%
\end{figure}


\section{Recording of the U/I-Curve}



 In the first experimental task the U/I-curve of the laser diode is recorded. This is done by biasing the diode with a current and measuring the voltage (cf. Figure \ref{fig:measurement1}). The bias current is increased from a minimum of 6.4~mA to 65~mA in steps of 2~mA. The resulting U/I-curve is shown in Figure \ref{fig:UI-curve}.

From
\begin{equation}
 \begin{split}
I = I\i{S0}\left[\e^{\beta\left(U-R\i{s}I\right)}-1\right] \hspace{2cm} I\leq I\i{S}\\
U = W\i{G}/e +R\i{S}I \hspace{2cm} I\geq I\i{S}\\
 \end{split}
\end{equation}
follows for the current above threshold:
\begin{equation}
 I = \frac{W\i{G}/e}{R\i{S}}-\frac{U}{R\i{S}} = a + b\cdot U
\end{equation}
The coefficients a and b can be determined by fitting a linear function to the U/I-curve above the threshold. The threshold current is calculated later in section \ref{ch:threshold}.  With these coefficients the band gap and the serial resistance of the laser diode can be calculated as:
\begin{equation}
\begin{split}
 W\i{G}/e=\\
 R\i{S} = 
\end{split}
\end{equation}


\section{P/I-Curve}
   