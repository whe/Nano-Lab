\chapter{Preparation}


\section{Exercise 1}
\footnote[1]{Freude, W; Scriptum - Optische Empf�nger und Fehlerwahrscheinlichkeit, 2009}The Bit Error Probability (BER) is given by

\begin{equation}
BER = \frac{1}{4}\mathrm{erfc}\left(\frac{u\i{s}-u_0}{\sqrt{2}\sigma_0}\right)+\frac{1}{4}\mathrm{erfc}\left(\frac{u\i{1}-u\i{s}}{\sqrt{2}\sigma_1}\right)
\label{eq:}
\end{equation}
with the complementary error function

\begin{equation}
\mathrm{erfc}(x)=\frac{2}{\sqrt{\pi}}\int^{\infty}_x\mathrm{e}^{-t^2}\mathrm{d}t\qquad,
\label{eq:}
\end{equation}
the decision threshold level $u\i{s}$ and the mean values $u_1$ of the mark and $u_0$ of the space.

To obtain the minimum BER the decision threshold is set that 1s and 0s occur with equal probability. In this case for a Gaussian probability distribution and a identical probability for the sending of marks and spaces the BER can be expressed by

\begin{equation}
BER = \frac{1}{2}\mathrm{erfc}\left(\frac{Q}{\sqrt{2}}\right)\qquad.
\label{eq:p1_BER}
\end{equation}

% <<<<<<< HEAD
% Using this equation the $BER$ is calculated for some values of $Q$ in Table \ref{tab:BER_Q}.
% \begin{table}[h]
%  \caption{$BER$ for some $Q$-Factors}
% \begin{center}
% \begin{tabular}{lrrrrrr}
% \toprule
% Q-Factor & 4.0 & 5.0 & 6.0 & 7.0 & 8.0 & 9.0\\
% \midrule
% $BER$ & $3.17\cdot 10^{-5}$ & $2.87\cdot 10^{-7}$ & $9.87\cdot 10^{-10}$ & $1.28\cdot 10^{-12}$ & $6.22\cdot 10^{-16}$ & $1.13\cdot 10^{-19}$\\
% \bottomrule
% \end{tabular}
% \end{center}
% \label{tab:BER_Q}
% \end{table}
% \todo{die Tabelle gef�llt mir noch net so}
% 
% In the case of a gausian distribution of the Signal the signal-to-noise ratio (SNR) can be calculated from the BER by \comseb{evtl ist das SNR einfach Q**2}
% 
% 
% \begin{table}[h]
%  \caption{$BER$ for some $Q$-Factors}
% \begin{center}
% \begin{tabular}{lrrrr}
% \toprule
% $BER$ & $10\E{-9}$ & $10\E{-11}$ & $10\E{-13}$ & $10\E{-15}$\\
% \midrule
% $SNR$ &&&&\\
% $t\i{err}$&&&&\\
% 
% \bottomrule
% \end{tabular}
% \end{center}
% \label{tab:BER_Q}
% \end{table}
% 
% 
% % 
% % \begin{equation}
% % \begin{split}
% % BER(Q=4) = \frac{1}{2}\mathrm{erfc}\left(\frac{4}{\sqrt{2}}\right)\\
% % =0.00003167
% % \end{split}
% % \label{eq:}
% % \end{equation}
% =======
Using this equation $BER$ can be calculated.

\begin{equation}
\begin{split}
BER(Q=4) = \frac{1}{2}\mathrm{erfc}\left(\frac{4}{\sqrt{2}}\right)\\
=0.00003167=3.167\cdot 10^{-5}
\end{split}
\label{eq:}
\end{equation}
% >>>>>>> 079253752d669474feb075f023471554faf5a2cb

\begin{table}[h]%
\centering
\caption{Minimum Bit Error Probabilities of different Q-factors}
 
\begin{tabular}{cc}

\toprule

$Q$	& $BER$\\
\midrule
4.0 & 3.167$\cdot 10^{-5}$\\
5.0& 2.867$\cdot 10^{-7}$  \\
6.0& 9.866$\cdot 10^{-10}$ \\
7.0& 1.280$\cdot 10^{-12}$ \\
8.0&6.221$\cdot 10^{-16}$ \\
9.0&1.129$\cdot 10^{-19}$ \\
\bottomrule 
\end{tabular}
\label{tab:1_daempfung}
\end{table}

There is no unique relationship between the signal-to-noise ration (SNR) and the BER. But considering several asumptions a relationship can be found. 

The probability density needs to be determined completely to the second momentum. This is the case for the Gaussian distribution. For the relationship 
\begin{equation}
1 < \frac{\sigma_1^2}{\sigma_0^2}\leq 2
\label{eq:}
\end{equation}
and pulse shapes like cos$^2(\pi t/T_t)$ there is a practical relationship between the SNR and Q-factor.
\begin{equation}
\gamma = Q^2
\label{eq:}
\end{equation}

\begin{table}[h]%
\centering
\caption{Sinal-to-nose ratios for different $Q$-factors.}
 
\begin{tabular}{cc}

\toprule

$BER$	& $\gamma$\\
\midrule
$10^{-9}$ &$10^{-18}$ \\
$10^{-11}$ &$10^{-22}$ \\
$10^{-13}$ &$10^{-26}$ \\
$10^{-15}$ &$10^{-30}$ \\
\bottomrule 
\end{tabular}
\label{tab:p1_daempfung}
\end{table}

The BER is the number of bit errors per total number of bits. 
This means for the BER of $10^{-9}$ that 1 error occurs every $10^{9}$ send bits. Assuming a 622 Mbps bit rate the time until an error occurs.


\begin{equation}
t\i{error}=\frac{10^9 \mathrm{bit}}{622\cdot10^{6} \mathrm{bit/s}} = 1.61~\mathrm{s}
\label{eq:}
\end{equation}

To receive a statistically relevant result it is possible to receive as many bits as possible. But for the task a number of bit errors of 10 is chosen. 
This leads to approximately 16 seconds to wait until a statistically relevant result can be made.

Table \ref{tab:p1_time} shows the times to wait for the different $BER$s.

\begin{table}[h]%
\centering
\caption{Time to wait to receive a statistical relevant result.}
 
\begin{tabular}{cc}

\toprule

$BER$	& $time$\\
\midrule
$10^{-9}$ &16~s\\
$10^{-11}$ &2~min 41~s \\
$10^{-13}$ &4~h~28~min \\
$10^{-15}$ &18~d~15~h\\
\bottomrule 
\end{tabular}
\label{tab:p1_time}
\end{table}

\newpage
\section{Exercise 2}
When measuring with small optical powers the noise is dominated by the receiver circuit noise, the thermal noise. To reach the shot noise limit the optical power needs to be high. Therefore for high powers the shot noise is relevant. In the case of the shot noise limit the SNR increases lineary with $\bar{P\i{e}}$ and depends only on the quantum efficiency $\eta$.\footnote[2]{Koos, C.; Optical Sources and Detectors; Lecture Notes} So one should measure in the shot noise limited case.



\section{Exercise 3}
The transmission system consists of a 80~km long fiber with a loss of 0.25~dB/km. Thus the attenuation caused by the fiber is 20~dB. There is a splice every kilometer in the fiber, thus there are 79 splices. At the input and the output of the fiber there's a connector with a loss of 0.4~dB each. The transmitter has an output power of 1~dBm and the detector has a sensitivity of -32~dBm. With a power margin of 7~dB the losses in the transmission system have to fulfill the following relation:
% \begin{equation}
%  \bar{P}\i{tr} = \bar{P}\i{rec} + C\i{L} + M\i{S}
% \end{equation}
\begin{equation}
1\mathrm{~dBm} \stackrel{!}{=}  - 32\mathrm{~dBm} +(20\mathrm{~dB}+79\cdot X+2\cdot 0.4\mathrm{~dB}) + 7\mathrm{~dB}
\end{equation}
Thus the attenuation of a splice $X$ has to be smaller than 0.07~dB.

In another system the power margin should be calculated. The systems transmission length of 70~km and a fiber loss of 0.3~dB/km corresponds to a loss of 21~dB. There are connectors at the begining and the end of the transmission line which cause a loss of 0.5~dB each. With a splice every kilometer and a loss of 0.07~dB at each splice there is an attenuation of 4.83~dB caused by the splices. Through dispersion there is a penalty of 5.5~dB. The reciever has a sensitivity of -32~dBm and the sender an output power of 3~dBm.
The power Budget of the Systems is:
\begin{equation}
 3\mathrm{~dBm} = -32\mathrm{~dBm} + (21\mathrm{~dB}+2\cdot 0.5\mathrm{~dB}+ 4.83\mathrm{~dB} +5.5\mathrm{~dB}) + M\i{S}
\end{equation}
Thus the power margin of the system is 2.67~dB. Because of degradation of the components or other unpredictable events the penalty of the system can change during operation. Thus the power margin should be 6-8~dB to ensure a reliable operation of the system.


% The losses in the given transmission system are:
% \begin{itemize}
%  \item 20~dB for the attenuation loss in the fiber
%  \item 0.8~dB for the two connectors
%  \item 79 splices with unknown losses
% \end{itemize}
%  \todo{das is auch doof. ich bin grad zu m�de}
\newpage

\section{Exercise 4}
To check if an optical transmission system can operate using a RZ format the rise time of the signal has to be calculated. The total rise time of the system $T\i{r}$ is calculated by
\begin{equation}
 T\i{r}^2 = T\i{tr}^2+T\i{f}^2+T\i{rec}^2
\end{equation}
where $T\i{tr}$ is the rise time of the transmitter, $T\i{f}$ the delay caused by the group velocity dispersion and $T\i{rec}$ the rise time of the reciever. Thus the rise time can be calculated as:
\begin{equation}
 T\i{r} = \sqrt{\left(0.3\mathrm{~ns}\right)^2+ \left(\frac{3 \mathrm{~ps}}{\mathrm{km}\cdot \mathrm{nm}}\cdot3\mathrm{~nm}\cdot 60\mathrm{~km}\right)^2 +\left(0.35\mathrm{~ns}\right)^2} = 0.71\mathrm{~ns}
\end{equation}
With the transmission rate of 1~Gbps corresponding to a transmission Bandwith of $B=$~1GHz the time bandwith product is $T\i{r}\cdot B = 0.71$. For using the RZ format $T\i{r}\cdot B$ has to be smaller than 0.35. Thus the RZ-format can't be used. For using the NRZ format a time bandwith product $T\i{r}\cdot B \leq 0.7$ is needed. Thus this format can't be used either. To use the given transmission system, the datarate has to be reduced. 


