\chapter{Preparation}


\section{Exercise 1}
The Bit Error Probability (BER) is given by

\begin{equation}
BER = \frac{1}{4}\mathrm{erfc}\left(\frac{u\i{s}-u_0}{\sqrt{2}\sigma_0}\right)+\frac{1}{4}\mathrm{erfc}\left(\frac{u\i{1}-u\i{s}}{\sqrt{2}\sigma_1}\right)
\label{eq:}
\end{equation}
with the complementary error function

\begin{equation}
\mathrm{erfc}(x)=\frac{2}{\sqrt{\pi}}\int^{\infty}_x\mathrm{e}^{-t^2}\mathrm{d}t\qquad,
\label{eq:}
\end{equation}
the decision threshold level $u\i{s}$ and the mean values $u_1$ of the mark and $u_0$ of the space.

To obtain the minimum BER the decision threshold is set that 1s and 0s occur with equal probability. In this case for a Gaussian probability distribution and a identical probability for the sending of marks and spaces the BER can be expressed by

\begin{equation}
BER = \frac{1}{2}\mathrm{erfc}\left(\frac{Q}{\sqrt{2}}\right)\qquad.
\label{eq:p1_BER}
\end{equation}

Using this equation $BER$ can be calculated.

\begin{equation}
\begin{split}
BER(Q=4) = \frac{1}{2}\mathrm{erfc}\left(\frac{4}{\sqrt{2}}\right)\\
=0.00003167=3.167\cdot 10^{-5}
\end{split}
\label{eq:}
\end{equation}

\begin{table}[h]%
\centering
\caption{Minimum Bit Error Probabilities of different Q-factors}
 
\begin{tabular}{cc}

\toprule

$Q$	& $BER$\\
\midrule
4.0 & 3.167$\cdot 10^{-5}$\\
5.0& 2.867$\cdot 10^{-7}$  \\
6.0& 9.866$\cdot 10^{-10}$ \\
7.0& 1.280$\cdot 10^{-12}$ \\
8.0&6.221$\cdot 10^{-16}$ \\
9.0&1.129$\cdot 10^{-19}$ \\
\bottomrule 
\end{tabular}
\label{tab:1_daempfung}
\end{table}

There is no unique relationship between the signal-to-noise ration (SNR) and the BER. But considering several asumptions a relationship can be found. 

The probability density needs to be determined completely to the second momentum. This is the case for the Gaussian distribution. For the relationship 
\begin{equation}
1 < \frac{\sigma_1^2}{\sigma_0^2}\leq 2
\label{eq:}
\end{equation}
and pulse shapes like cos$^2(\pi t/T_t)$ there is a practical relationship between the SNR and Q-factor.

\begin{equation}
\gamma = Q^2
\label{eq:}
\end{equation}

\begin{table}[h]%
\centering
\caption{Sinal-to-nose ratios for different $Q$-factors.}
 
\begin{tabular}{cc}

\toprule

$BER$	& $\gamma$\\
\midrule
$10^{-9}$ &$10^{-18}$ \\
$10^{-11}$ &$10^{-22}$ \\
$10^{-13}$ &$10^{-26}$ \\
$10^{-15}$ &$10^{-30}$ \\
\bottomrule 
\end{tabular}
\label{tab:1_daempfung}
\end{table}

Since the BER is the number of bit errors per total number of bits. 
This means for the BER of $10^{-9}$ that 1 error occurs every $10^{9}$ send bits. Assuming a 622 Mbps bit rate the time until an error occurs.

\begin{equation}
t\i{error}=\frac{10^9 bit}{622\cdot10^{6} bit/s} = 1.61~s
\label{eq:}
\end{equation}


\section{Exercise 2}




\section{Exercise 3}



\section{Exercise 4}