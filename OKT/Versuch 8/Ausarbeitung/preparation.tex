\chapter{Preparation}


\section{Exercise 1}
The Bit Error Probability (BER) is given by

\begin{equation}
BER = \frac{1}{4}\mathrm{erfc}\left(\frac{u\i{s}-u_0}{\sqrt{2}\sigma_0}\right)+\frac{1}{4}\mathrm{erfc}\left(\frac{u\i{1}-u\i{s}}{\sqrt{2}\sigma_1}\right)
\label{eq:}
\end{equation}
with the complementary error function

\begin{equation}
\mathrm{erfc}(x)=\frac{2}{\sqrt{\pi}}\int^{\infty}_x\mathrm{e}^{-t^2}\mathrm{d}t\qquad,
\label{eq:}
\end{equation}
the decision threshold level $u\i{s}$ and the mean values $u_1$ of the mark and $u_0$ of the space.

To obtain the minimum BER the decision threshold is set that 1s and 0s occur with equal probability. In this case for a Gaussian probability distribution and a identical probability for the sending of marks and spaces the BER can be expressed by

\begin{equation}
BER = \frac{1}{2}\mathrm{erfc}\left(\frac{Q}{\sqrt{2}}\right)\qquad.
\label{eq:p1_BER}
\end{equation}

Using this equation the $BER$ is calculated for some values of $Q$ in Table \ref{tab:BER_Q}.
\begin{table}[h]
 \caption{$BER$ for some $Q$-Factors}
\begin{center}
\begin{tabular}{lrrrrrr}
\toprule
Q-Factor & 4.0 & 5.0 & 6.0 & 7.0 & 8.0 & 9.0\\
\midrule
$BER$ & $3.17\cdot 10^{-5}$ & $2.87\cdot 10^{-7}$ & $9.87\cdot 10^{-10}$ & $1.28\cdot 10^{-12}$ & $6.22\cdot 10^{-16}$ & $1.13\cdot 10^{-19}$\\
\bottomrule
\end{tabular}
\end{center}
\label{tab:BER_Q}
\end{table}
\todo{die Tabelle gef�llt mir noch net so}

In the case of a gausian distribution of the Signal the signal-to-noise ratio (SNR) can be calculated from the BER by \comseb{evtl ist das SNR einfach Q**2}


\begin{table}[h]
 \caption{$BER$ for some $Q$-Factors}
\begin{center}
\begin{tabular}{lrrrr}
\toprule
$BER$ & $10\E{-9}$ & $10\E{-11}$ & $10\E{-13}$ & $10\E{-15}$\\
\midrule
$SNR$ &&&&\\
$t\i{err}$&&&&\\

\bottomrule
\end{tabular}
\end{center}
\label{tab:BER_Q}
\end{table}


% 
% \begin{equation}
% \begin{split}
% BER(Q=4) = \frac{1}{2}\mathrm{erfc}\left(\frac{4}{\sqrt{2}}\right)\\
% =0.00003167
% \end{split}
% \label{eq:}
% \end{equation}

\section{Exercise 2}




\section{Exercise 3}
The losses in the given transmission system are:
\begin{itemize}
 \item 20~dB for the attenuation loss in the fiber
 \item 0.8~dB for the two connectors
 \item 79 splices with unknown losses
\end{itemize}
 \todo{das is auch doof. ich bin grad zu m�de}


\section{Exercise 4}