\chapter{Preparation}


\section{Exercise 1}
The Bit Error Probability (BER) is given by

\begin{equation}
BER = \frac{1}{4}\mathrm{erfc}\left(\frac{u\i{s}-u_0}{\sqrt{2}\sigma_0}\right)+\frac{1}{4}\mathrm{erfc}\left(\frac{u\i{1}-u\i{s}}{\sqrt{2}\sigma_1}\right)
\label{eq:}
\end{equation}
with the complementary error function

\begin{equation}
\mathrm{erfc}(x)=\frac{2}{\sqrt{\pi}}\int^{\infty}_x\mathrm{e}^{-t^2}\mathrm{d}t\qquad,
\label{eq:}
\end{equation}
the decision threshold level $u\i{s}$ and the mean values $u_1$ of the mark and $u_0$ of the space.

To obtain the minimum BER the decision threshold is set that 1s and 0s occur with equal probability. In this case for a Gaussian probability distribution and a identical probability for the sending of marks and spaces the BER can be expressed by

\begin{equation}
BER = \frac{1}{2}\mathrm{erfc}\left(\frac{Q}{\sqrt{2}}\right)\qquad.
\label{eq:p1_BER}
\end{equation}

Using this equation $BER$ can be calculated.

\begin{equation}
\begin{split}
BER(Q=4) = \frac{1}{2}\mathrm{erfc}\left(\frac{4}{\sqrt{2}}\right)\\
=0.00003167
\end{split}
\label{eq:}
\end{equation}

\section{Exercise 2}




\section{Exercise 3}



\section{Exercise 4}