\chapter{Preparation}
\section{The Mach-Zehnder Modulator}
\begin{figure}
  \centering
  \includegraphics[width=.5\columnwidth]{Grafiken/Mach-Zehnder.jpg}

\caption{}
\label{fig:MZI}
\end{figure}
In a Mach-Zehnder Modulator (MZM) the light is split up in two branches. In each branch there is a non-linear medium, through which the Phase of the Light can be shifted. At the end the Light is brought together, so that it is interfering. This setup is shown in Figure \ref{fig:MZI}. The Amplitude of the Field at the end of the Modulator can be expressed as:
\begin{equation}
 E_{\mathrm{out}}=\exp\left(j\frac{\vartheta_1+\vartheta_2}{2}+j\frac{\vartheta_{\mathrm{Bias}}}{2} \right)\cdot\cos\left(\frac{\vartheta_1-\vartheta_2}{2}+\frac{\vartheta_{\mathrm{Bias}}}{2}\right)\cdot E_{\mathrm{in}} .
\end{equation}
The phase shift of the Signal at the output of the modulator is described by the first term, the amplitude by the second term. For the phase Modulation $\vartheta_1 = \vartheta_2$ only the phase of the Signal is changed while the Amplitude stays constant. This operation mode is called "`push-push"' mode. For $\vartheta_1 = -\vartheta_2$ only the Amplitude of the Signal is modulated. This operation mode is called "`push-pull"' mode. \footnote[1]{Leuthold, J. : Optical Communication Systems. WS 2010/2011}


\begin{figure}
  \centering
  \includegraphics[width=.5\columnwidth]{Grafiken/Mach-Zender-Transfer.jpg}

\caption{}
\label{fig:MZI_plot}
\end{figure}


\section{Modulation Formats}
\subsection{Amplitude Shift Keying}
In Amplitude Shift Keying (AFK) information is transmitted by amplitude of a signal. The level of the signal amplitude defines a binary value. There is the case "On/Off keying" (OOK) where the binary symbol is "0" when there is no power transmitted. It's a "1" when there is power transmitted. 
Figure \ref{fig:ask}a) shows such an ASK signal.\footnote[1]{Leuthold, J. : Optical Communication Systems. WS 2010/2011}

\begin{figure}
  \centering
  \includegraphics[width=.5\columnwidth]{Grafiken/OOK.jpg}
	\includegraphics[width=.5\columnwidth]{Grafiken/PSK.jpg}%
\caption{\textbf{a)} Amplitude Shift Keying \textbf{b)} Phase Shift Keying}
\label{fig:ask}
\end{figure}



\subsection{Phase Shift Keying}
In Phase Shift Keying (PSK) information is transmitted by modulating the phase of a carrier wave. Because the amplitude and the freuency of a carrier wave is not changed, the intensity of the signal is constant and thus not influenced by non-linear effects. An example for a psk modulated signal is shown in Figure \ref{fig:ask}b).\footnotemark[1]
In Quadrature Amplitude Modulation (QAM) both the phase and the amplitude of a carrier wave is modulated. Thus it is possible to transmit multiple bits per symbol. The QAM signal is generated by summing multiple amplitude modulated signals that have a defined phase offset. For Multilevel modulation the number of symbols is usualy given (e.g. 16-QAM). In principle it is also possible to implement multiple Symbols with PSK and ASK, too. Note that there is no difference between QPSK and 4-QAM.\footnotemark[1]
\section{Signal Generation}
\begin{figure}
  \centering
  \includegraphics[width=.5\columnwidth]{Grafiken/Signal-generation.jpg}

\caption{}
\label{fig:signal}
\end{figure}
\subsection{Generation of a OOK-Signal}
A OOK-Signal for high-speed telecommunication systems is commonly created through external modulation of a MZM (cf. Figure \ref{fig:signal}). To do so the MZM is used in the push-pull operation mode. The resulting signal can be described as the superposition of the carrier wave and the modulating signal (cf. figure \ref{fig:signal_sup}). \footnotemark[1]
\subsection{Generation of a BPSK-Signal}
For the generation of a BPSK the same setup as for the OOK-Signal can be used. The MZM is also operated in the push-pull mode, but with twice the switching Voltage used for OOK modulation. \todo{noch die Bilder irgendwie} \footnotemark[1]

\section{RZ Signal Generation}
For the generation of a RZ signal, first a NRZ signal is created for example with one of the methods described above. Then another MZM is used to carve the signal into a RZ shape. It is possible to realize different pulse widths for the RZ signal. When the MZM is driven by a sine with the 