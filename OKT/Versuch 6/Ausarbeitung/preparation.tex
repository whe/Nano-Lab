\chapter{Preparation}


\section{Design of an awesome resonator}
To filter a special wavelength using a ring resonator the design of the resonator needs to be adjusted to the wavelength.
\begin{figure}[h]%
\centering
\includegraphics[width=.5\columnwidth]{Grafiken/Resonator.png}%
\caption{Schematic of a ring resonator.}%
\label{fig:p1_ring}%
\end{figure} 
Figure \ref{fig:p1_ring}\footnote[1]{Jingshi Li, Materials for the preparation of Experiment 6} shows a schematic picture of a parallel ring resonator. It consists of a coupling zone between the ring resonator and the waveguide which can be modelled as directional coupler\footnotemark[1].

The transmission of the ring follows the relation:

\begin{equation}
a\i{r2}=a_{\mathrm{r1}}\cdot e^{-a/2\cdot L}\cdot e^{-j\beta L}
\label{eq:}
\end{equation}
That means the ring is in resonance
\begin{equation}
\beta L = m\cdot2\pi,\qquad m \in \mathbf{N}
\label{eq:}
\end{equation}
wiht $\beta = n_{\mathrm{eff}} \frac{2\pi f}{c} = n_{\mathrm{eff}} \frac{2\pi}{\lambda_0}$.

Using the transmission of the ring and the matrix of the directional coupler a term for the transmittet power can be derived\footnotemark[1]:
\begin{equation}
T(\beta L) = \frac{|a_t|^2}{|a_i|^2}= \frac{(1-e^{-a\cdot L})(1-(1-\kappa))}{(1-e^{-a/2\cdot L}\sqrt{1-\kappa})^2+4e^{-a/2\cdot L}\sqrt{1-\kappa}\cdot\mathrm{sin}^2(\beta L / 2)}}
\label{eq:}
\end{equation}
As described above the ring is in resonance for $\beta L = m\cdot2\pi$ and the transmission becomes minimum. 
In this case
\begin{equation}
T_{\mathrm{min}}=\frac{(e^{-a/2\cdot L} - \sqrt{1-\kappa})^2}{(1 - e^{-a/2\cdot L}\sqrt{1-\kappa})^2}
\label{eq:}
\end{equation}

When $e^{-a/2\cdot L} = \sqrt{1-\kappa}$ the transmission is zero. This case is calld $critical~coupling$ and can be achieved for $\a = - \frac{1}{L}\mathrm{ln}(1-\kappa)$.

To design ring resonator filter to block certain wavelength there are different approaches.
At first the filter could be consist different in series connected parallel ring resonators. Each 


\section{Measuring the Resonator Parameters}

\begin{figure}[h]%
\centering
\includegraphics[width=.5\columnwidth]{Grafiken/S21.pdf}%
\caption{Example Transmission Coefficient of a Resonator}%
\label{fig:S21}%
\end{figure}
\todo{Bild evtl auf f anpassen}

To caracterize a resonator its power transmission in dependency of the frequency can easily be measured. By that measurement, the whidth of the resonance lines at full width half maximum (FWHM) $\delta f$ and the free spectral range $\Delta f$ can determined. (cf. figure \ref{fig:S21}). The quotient $F= \Delta f/\delta f$ is called Finesse. For the case of critical coupling $F$ is given as:
\begin{equation}
 F = \frac{\Delta f}{\delta f} = \frac{\pi\sqrt{1-\kappa}}{\kappa}=\frac{\pi\exp\left(-\alpha/2L\right)}{1-\exp\left(-\alpha/L\right)}
\end{equation}
This can be rearranged to:
\begin{equation}
 \kappa = 0.5 \pm \sqrt{0.25+F^2/\pi^2}
\end{equation}
and
\begin{equation}
\alpha = -\frac{\ln F}{L(\ln F+2\ln\pi)} 
\end{equation}
respectively.

\section{Over-Critical and Under-Critical coupling}
