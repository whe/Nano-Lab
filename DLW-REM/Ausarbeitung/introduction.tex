%% introduction.tex
%%

%% ==============================
\chapter{Einleitung}
\label{ch:Introduction}
%% ==============================
Die Strukturierung von Werkstoffen ist ein Problem, das die Menschheit schon seit Jahrtausenden besch"aftigt. Schon in der Steinzeit haben sich die Menschen damit beschtigt, wie man aus harten Steinen Werkzeuge formen kann, die das Leben im Alltag erleichtern. Auch in der Mikroelektronik spielt die Strukturierung eine wichtige Rolle. Durch die Miniaturisierung werden die Herstellungskosten gesenkt und es lassen sich schnellere und spaarsamere integrierte Schaltungen produzieren. Das in der Mikroelektronik am h"aufigsten verwendete Strukturierungsverfahren ist die optische Lithographie. Diese erm"oglicht eine schnelle kosteng"unstige Strukturierung mit einer Aufl"osung kleiner 100~nm. F"ur bestimmte Anwendungen ist es jedoch notwendig dreidimensionale Strukturen zu schreiben.

Ein Strukturierungsverfahren das die Herstellung dreidimensionaler Strukturen erm�glicht ist das \textit{Direct Laser Writing} (DLW). Um die geschriebenen Strukturen zu charakterisieren kann ein Raster Elektronen Mikropskop (REM) genutzt werden.

Mit Hilfe eines REMs ist es m�glich, Strukturen im Mikro- und Nanometerbereich aufzul�sen und zu betrachten. Daher ist es ein Standardwerkzeug in vielen wissenschaftlichen Arbeitsfeldern.  Sowohl das Schreiben von Strukturen mit DLW als auch die Charakterisierung mittels eines REMs werden am Lichttechnischen Institut am Karlsruher Institut f�r Technologie durchgef�hrt. 

Im Versuch "`Direct Laser Writing und Raster Elektronenmikroskopie"' des Labors Nanoelektronik am Lichttechnischen Institut  wird daher ein  Einblick in das Arbeiten mit dem Raster Elektronen Mikroskop und das Herstellen dreidimensionaler Strukturen im Mikrometerbereich mit \textit{Direct Laser Writing} gegeben. 

%Ein m"ogliches neues Strukturierungsverfahren, das ebenfalls Strukturgr"osen mit hoher Aufl"osung erm"oglicht, ist die Laserinterferenzlithografie. 

%  Bis Heute haben sich zwar die verwendeten Materialien ge"andert, es sind jedoch immernoch Leistungsf"ahige Strukturierungsverfahren f"ur diese Materialien erforderlich. Auch die Anforderungen an die Materialien haben sich ge"andert. In der Mikroelektronik  
% 
%  als \todo{ich wills jetzt net zu arg breittreten. ma k"onnt noch was "uber die Strukturierung von anderen Materialien erz"ahlen: Metall gie"sen - schmieden - usw}.
% Auch in der Neuzeit besch"aftigen sich zahlreiche Wissenschaftler und Ingineure damit wie man Werkstoffe strukturieren kann. F"ur 