%% introduction.tex
%%

%% ==============================
\chapter{Einleitung}
\label{ch:Introduction}
%% ==============================
Die Strukturierung von Werkstoffen ist ein Problem, das die Menschheit schon seit Jahrtausenden besch"aftigt. Schon in der Steinzeit haben sich die Menschen damit beschtigt, wie man aus harten Steinen Werkzeuge Formen kann, die das Leben im Alltag erleichtern. Auch in der Mikroelektronik spielt die Strukturierung eine wichtige Rolle. Durch die Miniaturisierung werden die Herstellungskosten gesenkt und es lassen sich schnellere und spaarsamere integrierte Schaltungen produzieren. Das in der Mikroelektronik am h"aufigsten verwendete Strukturierungsverfahren ist die optische Lithographie. Diese erm"oglicht eine schnelle kosteng"unstige Strukturierung mit einer Aufl"osung kleiner 100~nm. F"ur bestimmte Anwendungen ist es jedoch notwendig dreidimensionale Strukturen zu schreiben.

Ein Strukturierungsverfahren das die Herstellung dreidimensionaler Strukturen erm�glicht ist das D\textit{irect Laser Writing}. Dieses Verfahren wird auch am Lichttechnischen Institut am Karlsruher Institut f�r Technologie genutzt.

Mit Hilfe eines Raster Elektronen Mikroskops ist es m�glich, Strukturen im Mikro- und Nanometerbereich aufzul�sen und zu betrachten. Darum ist es ein Standardwerkzeug in vielen wissenschaftlichen Arbeitsfeldern. 

Im Versuch "`Direct Laser Writing und Raster Elektronenmikroskopie"' des Labors Nanoelektronik am Lichttechnischen Institut soll zum einen das Arbeiten mit dem Raster Elektronen Mikroskop sowie das Herstellen von dreidimensionaler Strukturen im Mikrometerbereich mit \textit{Direct Laser Writing}. 

%Ein m"ogliches neues Strukturierungsverfahren, das ebenfalls Strukturgr"osen mit hoher Aufl"osung erm"oglicht, ist die Laserinterferenzlithografie. 

\todo{REM}


%  Bis Heute haben sich zwar die verwendeten Materialien ge"andert, es sind jedoch immernoch Leistungsf"ahige Strukturierungsverfahren f"ur diese Materialien erforderlich. Auch die Anforderungen an die Materialien haben sich ge"andert. In der Mikroelektronik  
% 
%  als \todo{ich wills jetzt net zu arg breittreten. ma k"onnt noch was "uber die Strukturierung von anderen Materialien erz"ahlen: Metall gie"sen - schmieden - usw}.
% Auch in der Neuzeit besch"aftigen sich zahlreiche Wissenschaftler und Ingineure damit wie man Werkstoffe strukturieren kann. F"ur 